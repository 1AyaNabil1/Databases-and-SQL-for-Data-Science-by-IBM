\documentclass[11pt]{article}

    \usepackage[breakable]{tcolorbox}
    \usepackage{parskip} % Stop auto-indenting (to mimic markdown behaviour)
    
    \usepackage{iftex}
    \ifPDFTeX
    	\usepackage[T1]{fontenc}
    	\usepackage{mathpazo}
    \else
    	\usepackage{fontspec}
    \fi

    % Basic figure setup, for now with no caption control since it's done
    % automatically by Pandoc (which extracts ![](path) syntax from Markdown).
    \usepackage{graphicx}
    % Maintain compatibility with old templates. Remove in nbconvert 6.0
    \let\Oldincludegraphics\includegraphics
    % Ensure that by default, figures have no caption (until we provide a
    % proper Figure object with a Caption API and a way to capture that
    % in the conversion process - todo).
    \usepackage{caption}
    \DeclareCaptionFormat{nocaption}{}
    \captionsetup{format=nocaption,aboveskip=0pt,belowskip=0pt}

    \usepackage[Export]{adjustbox} % Used to constrain images to a maximum size
    \adjustboxset{max size={0.9\linewidth}{0.9\paperheight}}
    \usepackage{float}
    \floatplacement{figure}{H} % forces figures to be placed at the correct location
    \usepackage{xcolor} % Allow colors to be defined
    \usepackage{enumerate} % Needed for markdown enumerations to work
    \usepackage{geometry} % Used to adjust the document margins
    \usepackage{amsmath} % Equations
    \usepackage{amssymb} % Equations
    \usepackage{textcomp} % defines textquotesingle
    % Hack from http://tex.stackexchange.com/a/47451/13684:
    \AtBeginDocument{%
        \def\PYZsq{\textquotesingle}% Upright quotes in Pygmentized code
    }
    \usepackage{upquote} % Upright quotes for verbatim code
    \usepackage{eurosym} % defines \euro
    \usepackage[mathletters]{ucs} % Extended unicode (utf-8) support
    \usepackage{fancyvrb} % verbatim replacement that allows latex
    \usepackage{grffile} % extends the file name processing of package graphics 
                         % to support a larger range
    \makeatletter % fix for grffile with XeLaTeX
    \def\Gread@@xetex#1{%
      \IfFileExists{"\Gin@base".bb}%
      {\Gread@eps{\Gin@base.bb}}%
      {\Gread@@xetex@aux#1}%
    }
    \makeatother

    % The hyperref package gives us a pdf with properly built
    % internal navigation ('pdf bookmarks' for the table of contents,
    % internal cross-reference links, web links for URLs, etc.)
    \usepackage{hyperref}
    % The default LaTeX title has an obnoxious amount of whitespace. By default,
    % titling removes some of it. It also provides customization options.
    \usepackage{titling}
    \usepackage{longtable} % longtable support required by pandoc >1.10
    \usepackage{booktabs}  % table support for pandoc > 1.12.2
    \usepackage[inline]{enumitem} % IRkernel/repr support (it uses the enumerate* environment)
    \usepackage[normalem]{ulem} % ulem is needed to support strikethroughs (\sout)
                                % normalem makes italics be italics, not underlines
    \usepackage{mathrsfs}
    

    
    % Colors for the hyperref package
    \definecolor{urlcolor}{rgb}{0,.145,.698}
    \definecolor{linkcolor}{rgb}{.71,0.21,0.01}
    \definecolor{citecolor}{rgb}{.12,.54,.11}

    % ANSI colors
    \definecolor{ansi-black}{HTML}{3E424D}
    \definecolor{ansi-black-intense}{HTML}{282C36}
    \definecolor{ansi-red}{HTML}{E75C58}
    \definecolor{ansi-red-intense}{HTML}{B22B31}
    \definecolor{ansi-green}{HTML}{00A250}
    \definecolor{ansi-green-intense}{HTML}{007427}
    \definecolor{ansi-yellow}{HTML}{DDB62B}
    \definecolor{ansi-yellow-intense}{HTML}{B27D12}
    \definecolor{ansi-blue}{HTML}{208FFB}
    \definecolor{ansi-blue-intense}{HTML}{0065CA}
    \definecolor{ansi-magenta}{HTML}{D160C4}
    \definecolor{ansi-magenta-intense}{HTML}{A03196}
    \definecolor{ansi-cyan}{HTML}{60C6C8}
    \definecolor{ansi-cyan-intense}{HTML}{258F8F}
    \definecolor{ansi-white}{HTML}{C5C1B4}
    \definecolor{ansi-white-intense}{HTML}{A1A6B2}
    \definecolor{ansi-default-inverse-fg}{HTML}{FFFFFF}
    \definecolor{ansi-default-inverse-bg}{HTML}{000000}

    % commands and environments needed by pandoc snippets
    % extracted from the output of `pandoc -s`
    \providecommand{\tightlist}{%
      \setlength{\itemsep}{0pt}\setlength{\parskip}{0pt}}
    \DefineVerbatimEnvironment{Highlighting}{Verbatim}{commandchars=\\\{\}}
    % Add ',fontsize=\small' for more characters per line
    \newenvironment{Shaded}{}{}
    \newcommand{\KeywordTok}[1]{\textcolor[rgb]{0.00,0.44,0.13}{\textbf{{#1}}}}
    \newcommand{\DataTypeTok}[1]{\textcolor[rgb]{0.56,0.13,0.00}{{#1}}}
    \newcommand{\DecValTok}[1]{\textcolor[rgb]{0.25,0.63,0.44}{{#1}}}
    \newcommand{\BaseNTok}[1]{\textcolor[rgb]{0.25,0.63,0.44}{{#1}}}
    \newcommand{\FloatTok}[1]{\textcolor[rgb]{0.25,0.63,0.44}{{#1}}}
    \newcommand{\CharTok}[1]{\textcolor[rgb]{0.25,0.44,0.63}{{#1}}}
    \newcommand{\StringTok}[1]{\textcolor[rgb]{0.25,0.44,0.63}{{#1}}}
    \newcommand{\CommentTok}[1]{\textcolor[rgb]{0.38,0.63,0.69}{\textit{{#1}}}}
    \newcommand{\OtherTok}[1]{\textcolor[rgb]{0.00,0.44,0.13}{{#1}}}
    \newcommand{\AlertTok}[1]{\textcolor[rgb]{1.00,0.00,0.00}{\textbf{{#1}}}}
    \newcommand{\FunctionTok}[1]{\textcolor[rgb]{0.02,0.16,0.49}{{#1}}}
    \newcommand{\RegionMarkerTok}[1]{{#1}}
    \newcommand{\ErrorTok}[1]{\textcolor[rgb]{1.00,0.00,0.00}{\textbf{{#1}}}}
    \newcommand{\NormalTok}[1]{{#1}}
    
    % Additional commands for more recent versions of Pandoc
    \newcommand{\ConstantTok}[1]{\textcolor[rgb]{0.53,0.00,0.00}{{#1}}}
    \newcommand{\SpecialCharTok}[1]{\textcolor[rgb]{0.25,0.44,0.63}{{#1}}}
    \newcommand{\VerbatimStringTok}[1]{\textcolor[rgb]{0.25,0.44,0.63}{{#1}}}
    \newcommand{\SpecialStringTok}[1]{\textcolor[rgb]{0.73,0.40,0.53}{{#1}}}
    \newcommand{\ImportTok}[1]{{#1}}
    \newcommand{\DocumentationTok}[1]{\textcolor[rgb]{0.73,0.13,0.13}{\textit{{#1}}}}
    \newcommand{\AnnotationTok}[1]{\textcolor[rgb]{0.38,0.63,0.69}{\textbf{\textit{{#1}}}}}
    \newcommand{\CommentVarTok}[1]{\textcolor[rgb]{0.38,0.63,0.69}{\textbf{\textit{{#1}}}}}
    \newcommand{\VariableTok}[1]{\textcolor[rgb]{0.10,0.09,0.49}{{#1}}}
    \newcommand{\ControlFlowTok}[1]{\textcolor[rgb]{0.00,0.44,0.13}{\textbf{{#1}}}}
    \newcommand{\OperatorTok}[1]{\textcolor[rgb]{0.40,0.40,0.40}{{#1}}}
    \newcommand{\BuiltInTok}[1]{{#1}}
    \newcommand{\ExtensionTok}[1]{{#1}}
    \newcommand{\PreprocessorTok}[1]{\textcolor[rgb]{0.74,0.48,0.00}{{#1}}}
    \newcommand{\AttributeTok}[1]{\textcolor[rgb]{0.49,0.56,0.16}{{#1}}}
    \newcommand{\InformationTok}[1]{\textcolor[rgb]{0.38,0.63,0.69}{\textbf{\textit{{#1}}}}}
    \newcommand{\WarningTok}[1]{\textcolor[rgb]{0.38,0.63,0.69}{\textbf{\textit{{#1}}}}}
    
    
    % Define a nice break command that doesn't care if a line doesn't already
    % exist.
    \def\br{\hspace*{\fill} \\* }
    % Math Jax compatibility definitions
    \def\gt{>}
    \def\lt{<}
    \let\Oldtex\TeX
    \let\Oldlatex\LaTeX
    \renewcommand{\TeX}{\textrm{\Oldtex}}
    \renewcommand{\LaTeX}{\textrm{\Oldlatex}}
    % Document parameters
    % Document title
    \title{DB0201EN-Week3-1-2-Querying-v4-py}
    
    
    
    
    
% Pygments definitions
\makeatletter
\def\PY@reset{\let\PY@it=\relax \let\PY@bf=\relax%
    \let\PY@ul=\relax \let\PY@tc=\relax%
    \let\PY@bc=\relax \let\PY@ff=\relax}
\def\PY@tok#1{\csname PY@tok@#1\endcsname}
\def\PY@toks#1+{\ifx\relax#1\empty\else%
    \PY@tok{#1}\expandafter\PY@toks\fi}
\def\PY@do#1{\PY@bc{\PY@tc{\PY@ul{%
    \PY@it{\PY@bf{\PY@ff{#1}}}}}}}
\def\PY#1#2{\PY@reset\PY@toks#1+\relax+\PY@do{#2}}

\expandafter\def\csname PY@tok@w\endcsname{\def\PY@tc##1{\textcolor[rgb]{0.73,0.73,0.73}{##1}}}
\expandafter\def\csname PY@tok@c\endcsname{\let\PY@it=\textit\def\PY@tc##1{\textcolor[rgb]{0.25,0.50,0.50}{##1}}}
\expandafter\def\csname PY@tok@cp\endcsname{\def\PY@tc##1{\textcolor[rgb]{0.74,0.48,0.00}{##1}}}
\expandafter\def\csname PY@tok@k\endcsname{\let\PY@bf=\textbf\def\PY@tc##1{\textcolor[rgb]{0.00,0.50,0.00}{##1}}}
\expandafter\def\csname PY@tok@kp\endcsname{\def\PY@tc##1{\textcolor[rgb]{0.00,0.50,0.00}{##1}}}
\expandafter\def\csname PY@tok@kt\endcsname{\def\PY@tc##1{\textcolor[rgb]{0.69,0.00,0.25}{##1}}}
\expandafter\def\csname PY@tok@o\endcsname{\def\PY@tc##1{\textcolor[rgb]{0.40,0.40,0.40}{##1}}}
\expandafter\def\csname PY@tok@ow\endcsname{\let\PY@bf=\textbf\def\PY@tc##1{\textcolor[rgb]{0.67,0.13,1.00}{##1}}}
\expandafter\def\csname PY@tok@nb\endcsname{\def\PY@tc##1{\textcolor[rgb]{0.00,0.50,0.00}{##1}}}
\expandafter\def\csname PY@tok@nf\endcsname{\def\PY@tc##1{\textcolor[rgb]{0.00,0.00,1.00}{##1}}}
\expandafter\def\csname PY@tok@nc\endcsname{\let\PY@bf=\textbf\def\PY@tc##1{\textcolor[rgb]{0.00,0.00,1.00}{##1}}}
\expandafter\def\csname PY@tok@nn\endcsname{\let\PY@bf=\textbf\def\PY@tc##1{\textcolor[rgb]{0.00,0.00,1.00}{##1}}}
\expandafter\def\csname PY@tok@ne\endcsname{\let\PY@bf=\textbf\def\PY@tc##1{\textcolor[rgb]{0.82,0.25,0.23}{##1}}}
\expandafter\def\csname PY@tok@nv\endcsname{\def\PY@tc##1{\textcolor[rgb]{0.10,0.09,0.49}{##1}}}
\expandafter\def\csname PY@tok@no\endcsname{\def\PY@tc##1{\textcolor[rgb]{0.53,0.00,0.00}{##1}}}
\expandafter\def\csname PY@tok@nl\endcsname{\def\PY@tc##1{\textcolor[rgb]{0.63,0.63,0.00}{##1}}}
\expandafter\def\csname PY@tok@ni\endcsname{\let\PY@bf=\textbf\def\PY@tc##1{\textcolor[rgb]{0.60,0.60,0.60}{##1}}}
\expandafter\def\csname PY@tok@na\endcsname{\def\PY@tc##1{\textcolor[rgb]{0.49,0.56,0.16}{##1}}}
\expandafter\def\csname PY@tok@nt\endcsname{\let\PY@bf=\textbf\def\PY@tc##1{\textcolor[rgb]{0.00,0.50,0.00}{##1}}}
\expandafter\def\csname PY@tok@nd\endcsname{\def\PY@tc##1{\textcolor[rgb]{0.67,0.13,1.00}{##1}}}
\expandafter\def\csname PY@tok@s\endcsname{\def\PY@tc##1{\textcolor[rgb]{0.73,0.13,0.13}{##1}}}
\expandafter\def\csname PY@tok@sd\endcsname{\let\PY@it=\textit\def\PY@tc##1{\textcolor[rgb]{0.73,0.13,0.13}{##1}}}
\expandafter\def\csname PY@tok@si\endcsname{\let\PY@bf=\textbf\def\PY@tc##1{\textcolor[rgb]{0.73,0.40,0.53}{##1}}}
\expandafter\def\csname PY@tok@se\endcsname{\let\PY@bf=\textbf\def\PY@tc##1{\textcolor[rgb]{0.73,0.40,0.13}{##1}}}
\expandafter\def\csname PY@tok@sr\endcsname{\def\PY@tc##1{\textcolor[rgb]{0.73,0.40,0.53}{##1}}}
\expandafter\def\csname PY@tok@ss\endcsname{\def\PY@tc##1{\textcolor[rgb]{0.10,0.09,0.49}{##1}}}
\expandafter\def\csname PY@tok@sx\endcsname{\def\PY@tc##1{\textcolor[rgb]{0.00,0.50,0.00}{##1}}}
\expandafter\def\csname PY@tok@m\endcsname{\def\PY@tc##1{\textcolor[rgb]{0.40,0.40,0.40}{##1}}}
\expandafter\def\csname PY@tok@gh\endcsname{\let\PY@bf=\textbf\def\PY@tc##1{\textcolor[rgb]{0.00,0.00,0.50}{##1}}}
\expandafter\def\csname PY@tok@gu\endcsname{\let\PY@bf=\textbf\def\PY@tc##1{\textcolor[rgb]{0.50,0.00,0.50}{##1}}}
\expandafter\def\csname PY@tok@gd\endcsname{\def\PY@tc##1{\textcolor[rgb]{0.63,0.00,0.00}{##1}}}
\expandafter\def\csname PY@tok@gi\endcsname{\def\PY@tc##1{\textcolor[rgb]{0.00,0.63,0.00}{##1}}}
\expandafter\def\csname PY@tok@gr\endcsname{\def\PY@tc##1{\textcolor[rgb]{1.00,0.00,0.00}{##1}}}
\expandafter\def\csname PY@tok@ge\endcsname{\let\PY@it=\textit}
\expandafter\def\csname PY@tok@gs\endcsname{\let\PY@bf=\textbf}
\expandafter\def\csname PY@tok@gp\endcsname{\let\PY@bf=\textbf\def\PY@tc##1{\textcolor[rgb]{0.00,0.00,0.50}{##1}}}
\expandafter\def\csname PY@tok@go\endcsname{\def\PY@tc##1{\textcolor[rgb]{0.53,0.53,0.53}{##1}}}
\expandafter\def\csname PY@tok@gt\endcsname{\def\PY@tc##1{\textcolor[rgb]{0.00,0.27,0.87}{##1}}}
\expandafter\def\csname PY@tok@err\endcsname{\def\PY@bc##1{\setlength{\fboxsep}{0pt}\fcolorbox[rgb]{1.00,0.00,0.00}{1,1,1}{\strut ##1}}}
\expandafter\def\csname PY@tok@kc\endcsname{\let\PY@bf=\textbf\def\PY@tc##1{\textcolor[rgb]{0.00,0.50,0.00}{##1}}}
\expandafter\def\csname PY@tok@kd\endcsname{\let\PY@bf=\textbf\def\PY@tc##1{\textcolor[rgb]{0.00,0.50,0.00}{##1}}}
\expandafter\def\csname PY@tok@kn\endcsname{\let\PY@bf=\textbf\def\PY@tc##1{\textcolor[rgb]{0.00,0.50,0.00}{##1}}}
\expandafter\def\csname PY@tok@kr\endcsname{\let\PY@bf=\textbf\def\PY@tc##1{\textcolor[rgb]{0.00,0.50,0.00}{##1}}}
\expandafter\def\csname PY@tok@bp\endcsname{\def\PY@tc##1{\textcolor[rgb]{0.00,0.50,0.00}{##1}}}
\expandafter\def\csname PY@tok@fm\endcsname{\def\PY@tc##1{\textcolor[rgb]{0.00,0.00,1.00}{##1}}}
\expandafter\def\csname PY@tok@vc\endcsname{\def\PY@tc##1{\textcolor[rgb]{0.10,0.09,0.49}{##1}}}
\expandafter\def\csname PY@tok@vg\endcsname{\def\PY@tc##1{\textcolor[rgb]{0.10,0.09,0.49}{##1}}}
\expandafter\def\csname PY@tok@vi\endcsname{\def\PY@tc##1{\textcolor[rgb]{0.10,0.09,0.49}{##1}}}
\expandafter\def\csname PY@tok@vm\endcsname{\def\PY@tc##1{\textcolor[rgb]{0.10,0.09,0.49}{##1}}}
\expandafter\def\csname PY@tok@sa\endcsname{\def\PY@tc##1{\textcolor[rgb]{0.73,0.13,0.13}{##1}}}
\expandafter\def\csname PY@tok@sb\endcsname{\def\PY@tc##1{\textcolor[rgb]{0.73,0.13,0.13}{##1}}}
\expandafter\def\csname PY@tok@sc\endcsname{\def\PY@tc##1{\textcolor[rgb]{0.73,0.13,0.13}{##1}}}
\expandafter\def\csname PY@tok@dl\endcsname{\def\PY@tc##1{\textcolor[rgb]{0.73,0.13,0.13}{##1}}}
\expandafter\def\csname PY@tok@s2\endcsname{\def\PY@tc##1{\textcolor[rgb]{0.73,0.13,0.13}{##1}}}
\expandafter\def\csname PY@tok@sh\endcsname{\def\PY@tc##1{\textcolor[rgb]{0.73,0.13,0.13}{##1}}}
\expandafter\def\csname PY@tok@s1\endcsname{\def\PY@tc##1{\textcolor[rgb]{0.73,0.13,0.13}{##1}}}
\expandafter\def\csname PY@tok@mb\endcsname{\def\PY@tc##1{\textcolor[rgb]{0.40,0.40,0.40}{##1}}}
\expandafter\def\csname PY@tok@mf\endcsname{\def\PY@tc##1{\textcolor[rgb]{0.40,0.40,0.40}{##1}}}
\expandafter\def\csname PY@tok@mh\endcsname{\def\PY@tc##1{\textcolor[rgb]{0.40,0.40,0.40}{##1}}}
\expandafter\def\csname PY@tok@mi\endcsname{\def\PY@tc##1{\textcolor[rgb]{0.40,0.40,0.40}{##1}}}
\expandafter\def\csname PY@tok@il\endcsname{\def\PY@tc##1{\textcolor[rgb]{0.40,0.40,0.40}{##1}}}
\expandafter\def\csname PY@tok@mo\endcsname{\def\PY@tc##1{\textcolor[rgb]{0.40,0.40,0.40}{##1}}}
\expandafter\def\csname PY@tok@ch\endcsname{\let\PY@it=\textit\def\PY@tc##1{\textcolor[rgb]{0.25,0.50,0.50}{##1}}}
\expandafter\def\csname PY@tok@cm\endcsname{\let\PY@it=\textit\def\PY@tc##1{\textcolor[rgb]{0.25,0.50,0.50}{##1}}}
\expandafter\def\csname PY@tok@cpf\endcsname{\let\PY@it=\textit\def\PY@tc##1{\textcolor[rgb]{0.25,0.50,0.50}{##1}}}
\expandafter\def\csname PY@tok@c1\endcsname{\let\PY@it=\textit\def\PY@tc##1{\textcolor[rgb]{0.25,0.50,0.50}{##1}}}
\expandafter\def\csname PY@tok@cs\endcsname{\let\PY@it=\textit\def\PY@tc##1{\textcolor[rgb]{0.25,0.50,0.50}{##1}}}

\def\PYZbs{\char`\\}
\def\PYZus{\char`\_}
\def\PYZob{\char`\{}
\def\PYZcb{\char`\}}
\def\PYZca{\char`\^}
\def\PYZam{\char`\&}
\def\PYZlt{\char`\<}
\def\PYZgt{\char`\>}
\def\PYZsh{\char`\#}
\def\PYZpc{\char`\%}
\def\PYZdl{\char`\$}
\def\PYZhy{\char`\-}
\def\PYZsq{\char`\'}
\def\PYZdq{\char`\"}
\def\PYZti{\char`\~}
% for compatibility with earlier versions
\def\PYZat{@}
\def\PYZlb{[}
\def\PYZrb{]}
\makeatother


    % For linebreaks inside Verbatim environment from package fancyvrb. 
    \makeatletter
        \newbox\Wrappedcontinuationbox 
        \newbox\Wrappedvisiblespacebox 
        \newcommand*\Wrappedvisiblespace {\textcolor{red}{\textvisiblespace}} 
        \newcommand*\Wrappedcontinuationsymbol {\textcolor{red}{\llap{\tiny$\m@th\hookrightarrow$}}} 
        \newcommand*\Wrappedcontinuationindent {3ex } 
        \newcommand*\Wrappedafterbreak {\kern\Wrappedcontinuationindent\copy\Wrappedcontinuationbox} 
        % Take advantage of the already applied Pygments mark-up to insert 
        % potential linebreaks for TeX processing. 
        %        {, <, #, %, $, ' and ": go to next line. 
        %        _, }, ^, &, >, - and ~: stay at end of broken line. 
        % Use of \textquotesingle for straight quote. 
        \newcommand*\Wrappedbreaksatspecials {% 
            \def\PYGZus{\discretionary{\char`\_}{\Wrappedafterbreak}{\char`\_}}% 
            \def\PYGZob{\discretionary{}{\Wrappedafterbreak\char`\{}{\char`\{}}% 
            \def\PYGZcb{\discretionary{\char`\}}{\Wrappedafterbreak}{\char`\}}}% 
            \def\PYGZca{\discretionary{\char`\^}{\Wrappedafterbreak}{\char`\^}}% 
            \def\PYGZam{\discretionary{\char`\&}{\Wrappedafterbreak}{\char`\&}}% 
            \def\PYGZlt{\discretionary{}{\Wrappedafterbreak\char`\<}{\char`\<}}% 
            \def\PYGZgt{\discretionary{\char`\>}{\Wrappedafterbreak}{\char`\>}}% 
            \def\PYGZsh{\discretionary{}{\Wrappedafterbreak\char`\#}{\char`\#}}% 
            \def\PYGZpc{\discretionary{}{\Wrappedafterbreak\char`\%}{\char`\%}}% 
            \def\PYGZdl{\discretionary{}{\Wrappedafterbreak\char`\$}{\char`\$}}% 
            \def\PYGZhy{\discretionary{\char`\-}{\Wrappedafterbreak}{\char`\-}}% 
            \def\PYGZsq{\discretionary{}{\Wrappedafterbreak\textquotesingle}{\textquotesingle}}% 
            \def\PYGZdq{\discretionary{}{\Wrappedafterbreak\char`\"}{\char`\"}}% 
            \def\PYGZti{\discretionary{\char`\~}{\Wrappedafterbreak}{\char`\~}}% 
        } 
        % Some characters . , ; ? ! / are not pygmentized. 
        % This macro makes them "active" and they will insert potential linebreaks 
        \newcommand*\Wrappedbreaksatpunct {% 
            \lccode`\~`\.\lowercase{\def~}{\discretionary{\hbox{\char`\.}}{\Wrappedafterbreak}{\hbox{\char`\.}}}% 
            \lccode`\~`\,\lowercase{\def~}{\discretionary{\hbox{\char`\,}}{\Wrappedafterbreak}{\hbox{\char`\,}}}% 
            \lccode`\~`\;\lowercase{\def~}{\discretionary{\hbox{\char`\;}}{\Wrappedafterbreak}{\hbox{\char`\;}}}% 
            \lccode`\~`\:\lowercase{\def~}{\discretionary{\hbox{\char`\:}}{\Wrappedafterbreak}{\hbox{\char`\:}}}% 
            \lccode`\~`\?\lowercase{\def~}{\discretionary{\hbox{\char`\?}}{\Wrappedafterbreak}{\hbox{\char`\?}}}% 
            \lccode`\~`\!\lowercase{\def~}{\discretionary{\hbox{\char`\!}}{\Wrappedafterbreak}{\hbox{\char`\!}}}% 
            \lccode`\~`\/\lowercase{\def~}{\discretionary{\hbox{\char`\/}}{\Wrappedafterbreak}{\hbox{\char`\/}}}% 
            \catcode`\.\active
            \catcode`\,\active 
            \catcode`\;\active
            \catcode`\:\active
            \catcode`\?\active
            \catcode`\!\active
            \catcode`\/\active 
            \lccode`\~`\~ 	
        }
    \makeatother

    \let\OriginalVerbatim=\Verbatim
    \makeatletter
    \renewcommand{\Verbatim}[1][1]{%
        %\parskip\z@skip
        \sbox\Wrappedcontinuationbox {\Wrappedcontinuationsymbol}%
        \sbox\Wrappedvisiblespacebox {\FV@SetupFont\Wrappedvisiblespace}%
        \def\FancyVerbFormatLine ##1{\hsize\linewidth
            \vtop{\raggedright\hyphenpenalty\z@\exhyphenpenalty\z@
                \doublehyphendemerits\z@\finalhyphendemerits\z@
                \strut ##1\strut}%
        }%
        % If the linebreak is at a space, the latter will be displayed as visible
        % space at end of first line, and a continuation symbol starts next line.
        % Stretch/shrink are however usually zero for typewriter font.
        \def\FV@Space {%
            \nobreak\hskip\z@ plus\fontdimen3\font minus\fontdimen4\font
            \discretionary{\copy\Wrappedvisiblespacebox}{\Wrappedafterbreak}
            {\kern\fontdimen2\font}%
        }%
        
        % Allow breaks at special characters using \PYG... macros.
        \Wrappedbreaksatspecials
        % Breaks at punctuation characters . , ; ? ! and / need catcode=\active 	
        \OriginalVerbatim[#1,codes*=\Wrappedbreaksatpunct]%
    }
    \makeatother

    % Exact colors from NB
    \definecolor{incolor}{HTML}{303F9F}
    \definecolor{outcolor}{HTML}{D84315}
    \definecolor{cellborder}{HTML}{CFCFCF}
    \definecolor{cellbackground}{HTML}{F7F7F7}
    
    % prompt
    \makeatletter
    \newcommand{\boxspacing}{\kern\kvtcb@left@rule\kern\kvtcb@boxsep}
    \makeatother
    \newcommand{\prompt}[4]{
        \ttfamily\llap{{\color{#2}[#3]:\hspace{3pt}#4}}\vspace{-\baselineskip}
    }
    

    
    % Prevent overflowing lines due to hard-to-break entities
    \sloppy 
    % Setup hyperref package
    \hypersetup{
      breaklinks=true,  % so long urls are correctly broken across lines
      colorlinks=true,
      urlcolor=urlcolor,
      linkcolor=linkcolor,
      citecolor=citecolor,
      }
    % Slightly bigger margins than the latex defaults
    
    \geometry{verbose,tmargin=1in,bmargin=1in,lmargin=1in,rmargin=1in}
    
    

\begin{document}
    
    \maketitle
    
    

    
    Lab: Access DB2 on Cloud using Python

    \hypertarget{introduction}{%
\section{Introduction}\label{introduction}}

This notebook illustrates how to access your database instance using
Python by following the steps below: 1. Import the \texttt{ibm\_db}
Python library 1. Identify and enter the database connection credentials
1. Create the database connection 1. Create a table 1. Insert data into
the table 1. Query data from the table 1. Retrieve the result set into a
pandas dataframe 1. Close the database connection

\textbf{Notice:} Please follow the instructions given in the first Lab
of this course to Create a database service instance of Db2 on Cloud.

\hypertarget{task-1-import-the-ibm_db-python-library}{%
\subsection{\texorpdfstring{Task 1: Import the \texttt{ibm\_db} Python
library}{Task 1: Import the ibm\_db Python library}}\label{task-1-import-the-ibm_db-python-library}}

The \texttt{ibm\_db} \href{https://pypi.python.org/pypi/ibm_db/}{API}
provides a variety of useful Python functions for accessing and
manipulating data in an IBM® data server database, including functions
for connecting to a database, preparing and issuing SQL statements,
fetching rows from result sets, calling stored procedures, committing
and rolling back transactions, handling errors, and retrieving metadata.

We import the ibm\_db library into our Python Application

    \begin{tcolorbox}[breakable, size=fbox, boxrule=1pt, pad at break*=1mm,colback=cellbackground, colframe=cellborder]
\prompt{In}{incolor}{1}{\boxspacing}
\begin{Verbatim}[commandchars=\\\{\}]
\PY{k+kn}{import} \PY{n+nn}{ibm\PYZus{}db}
\end{Verbatim}
\end{tcolorbox}

    When the command above completes, the \texttt{ibm\_db} library is loaded
in your notebook.

\hypertarget{task-2-identify-the-database-connection-credentials}{%
\subsection{Task 2: Identify the database connection
credentials}\label{task-2-identify-the-database-connection-credentials}}

Connecting to dashDB or DB2 database requires the following information:
* Driver Name * Database name * Host DNS name or IP address * Host port
* Connection protocol * User ID * User Password

\textbf{Notice:} To obtain credentials please refer to the instructions
given in the first Lab of this course

Now enter your database credentials below

Replace the placeholder values in angular brackets
\textless{}\textgreater{} below with your actual database credentials

e.g.~replace ``database'' with ``BLUDB''

    \begin{tcolorbox}[breakable, size=fbox, boxrule=1pt, pad at break*=1mm,colback=cellbackground, colframe=cellborder]
\prompt{In}{incolor}{5}{\boxspacing}
\begin{Verbatim}[commandchars=\\\{\}]
\PY{c+c1}{\PYZsh{}Replace the placeholder values with the actuals for your Db2 Service Credentials}
\PY{n}{dsn\PYZus{}driver} \PY{o}{=} \PY{l+s+s2}{\PYZdq{}}\PY{l+s+s2}{\PYZob{}}\PY{l+s+s2}{IBM DB2 ODBC DRIVER\PYZcb{}}\PY{l+s+s2}{\PYZdq{}}
\PY{n}{dsn\PYZus{}database} \PY{o}{=} \PY{l+s+s2}{\PYZdq{}}\PY{l+s+s2}{BLUDB}\PY{l+s+s2}{\PYZdq{}}             \PY{c+c1}{\PYZsh{} e.g. \PYZdq{}BLUDB\PYZdq{}}
\PY{n}{dsn\PYZus{}hostname} \PY{o}{=} \PY{l+s+s2}{\PYZdq{}}\PY{l+s+s2}{hostname}\PY{l+s+s2}{\PYZdq{}}          \PY{c+c1}{\PYZsh{} e.g.: \PYZdq{}dashdb\PYZhy{}txn\PYZhy{}sbox\PYZhy{}yp\PYZhy{}dal09\PYZhy{}04.services.dal.bluemix.net\PYZdq{}}
\PY{n}{dsn\PYZus{}port} \PY{o}{=} \PY{l+s+s2}{\PYZdq{}}\PY{l+s+s2}{50000}\PY{l+s+s2}{\PYZdq{}}                 \PY{c+c1}{\PYZsh{} e.g. \PYZdq{}50000\PYZdq{} }
\PY{n}{dsn\PYZus{}protocol} \PY{o}{=} \PY{l+s+s2}{\PYZdq{}}\PY{l+s+s2}{TCPIP}\PY{l+s+s2}{\PYZdq{}}             \PY{c+c1}{\PYZsh{} i.e. \PYZdq{}TCPIP\PYZdq{}}
\PY{n}{dsn\PYZus{}uid} \PY{o}{=} \PY{l+s+s2}{\PYZdq{}}\PY{l+s+s2}{username}\PY{l+s+s2}{\PYZdq{}}               \PY{c+c1}{\PYZsh{} e.g. \PYZdq{}abc12345\PYZdq{}}
\PY{n}{dsn\PYZus{}pwd} \PY{o}{=} \PY{l+s+s2}{\PYZdq{}}\PY{l+s+s2}{password}\PY{l+s+s2}{\PYZdq{}}               \PY{c+c1}{\PYZsh{} e.g. \PYZdq{}7dBZ3wWt9XN6\PYZdl{}o0J\PYZdq{}}
\end{Verbatim}
\end{tcolorbox}

    \hypertarget{task-3-create-the-database-connection}{%
\subsection{Task 3: Create the database
connection}\label{task-3-create-the-database-connection}}

Ibm\_db API uses the IBM Data Server Driver for ODBC and CLI APIs to
connect to IBM DB2 and Informix.

Create the database connection

    \begin{tcolorbox}[breakable, size=fbox, boxrule=1pt, pad at break*=1mm,colback=cellbackground, colframe=cellborder]
\prompt{In}{incolor}{6}{\boxspacing}
\begin{Verbatim}[commandchars=\\\{\}]
\PY{c+c1}{\PYZsh{}Create database connection}
\PY{c+c1}{\PYZsh{}DO NOT MODIFY THIS CELL. Just RUN it with Shift + Enter}
\PY{n}{dsn} \PY{o}{=} \PY{p}{(}
    \PY{l+s+s2}{\PYZdq{}}\PY{l+s+s2}{DRIVER=}\PY{l+s+si}{\PYZob{}0\PYZcb{}}\PY{l+s+s2}{;}\PY{l+s+s2}{\PYZdq{}}
    \PY{l+s+s2}{\PYZdq{}}\PY{l+s+s2}{DATABASE=}\PY{l+s+si}{\PYZob{}1\PYZcb{}}\PY{l+s+s2}{;}\PY{l+s+s2}{\PYZdq{}}
    \PY{l+s+s2}{\PYZdq{}}\PY{l+s+s2}{HOSTNAME=}\PY{l+s+si}{\PYZob{}2\PYZcb{}}\PY{l+s+s2}{;}\PY{l+s+s2}{\PYZdq{}}
    \PY{l+s+s2}{\PYZdq{}}\PY{l+s+s2}{PORT=}\PY{l+s+si}{\PYZob{}3\PYZcb{}}\PY{l+s+s2}{;}\PY{l+s+s2}{\PYZdq{}}
    \PY{l+s+s2}{\PYZdq{}}\PY{l+s+s2}{PROTOCOL=}\PY{l+s+si}{\PYZob{}4\PYZcb{}}\PY{l+s+s2}{;}\PY{l+s+s2}{\PYZdq{}}
    \PY{l+s+s2}{\PYZdq{}}\PY{l+s+s2}{UID=}\PY{l+s+si}{\PYZob{}5\PYZcb{}}\PY{l+s+s2}{;}\PY{l+s+s2}{\PYZdq{}}
    \PY{l+s+s2}{\PYZdq{}}\PY{l+s+s2}{PWD=}\PY{l+s+si}{\PYZob{}6\PYZcb{}}\PY{l+s+s2}{;}\PY{l+s+s2}{\PYZdq{}}\PY{p}{)}\PY{o}{.}\PY{n}{format}\PY{p}{(}\PY{n}{dsn\PYZus{}driver}\PY{p}{,} \PY{n}{dsn\PYZus{}database}\PY{p}{,} \PY{n}{dsn\PYZus{}hostname}\PY{p}{,} \PY{n}{dsn\PYZus{}port}\PY{p}{,} \PY{n}{dsn\PYZus{}protocol}\PY{p}{,} \PY{n}{dsn\PYZus{}uid}\PY{p}{,} \PY{n}{dsn\PYZus{}pwd}\PY{p}{)}

\PY{k}{try}\PY{p}{:}
    \PY{n}{conn} \PY{o}{=} \PY{n}{ibm\PYZus{}db}\PY{o}{.}\PY{n}{connect}\PY{p}{(}\PY{n}{dsn}\PY{p}{,} \PY{l+s+s2}{\PYZdq{}}\PY{l+s+s2}{\PYZdq{}}\PY{p}{,} \PY{l+s+s2}{\PYZdq{}}\PY{l+s+s2}{\PYZdq{}}\PY{p}{)}
    \PY{n+nb}{print} \PY{p}{(}\PY{l+s+s2}{\PYZdq{}}\PY{l+s+s2}{Connected to database: }\PY{l+s+s2}{\PYZdq{}}\PY{p}{,} \PY{n}{dsn\PYZus{}database}\PY{p}{,} \PY{l+s+s2}{\PYZdq{}}\PY{l+s+s2}{as user: }\PY{l+s+s2}{\PYZdq{}}\PY{p}{,} \PY{n}{dsn\PYZus{}uid}\PY{p}{,} \PY{l+s+s2}{\PYZdq{}}\PY{l+s+s2}{on host: }\PY{l+s+s2}{\PYZdq{}}\PY{p}{,} \PY{n}{dsn\PYZus{}hostname}\PY{p}{)}

\PY{k}{except}\PY{p}{:}
    \PY{n+nb}{print} \PY{p}{(}\PY{l+s+s2}{\PYZdq{}}\PY{l+s+s2}{Unable to connect: }\PY{l+s+s2}{\PYZdq{}}\PY{p}{,} \PY{n}{ibm\PYZus{}db}\PY{o}{.}\PY{n}{conn\PYZus{}errormsg}\PY{p}{(}\PY{p}{)} \PY{p}{)}
\end{Verbatim}
\end{tcolorbox}

    \begin{Verbatim}[commandchars=\\\{\}]
Connected to database:  BLUDB as user:  nrx71347 on host:  dashdb-txn-sbox-yp-
lon02-07.services.eu-gb.bluemix.net
    \end{Verbatim}

    \hypertarget{task-4-create-a-table-in-the-database}{%
\subsection{Task 4: Create a table in the
database}\label{task-4-create-a-table-in-the-database}}

In this step we will create a table in the database with following
details:

    \begin{tcolorbox}[breakable, size=fbox, boxrule=1pt, pad at break*=1mm,colback=cellbackground, colframe=cellborder]
\prompt{In}{incolor}{7}{\boxspacing}
\begin{Verbatim}[commandchars=\\\{\}]
\PY{c+c1}{\PYZsh{}Lets first drop the table INSTRUCTOR in case it exists from a previous attempt}
\PY{n}{dropQuery} \PY{o}{=} \PY{l+s+s2}{\PYZdq{}}\PY{l+s+s2}{drop table INSTRUCTOR}\PY{l+s+s2}{\PYZdq{}}

\PY{c+c1}{\PYZsh{}Now execute the drop statment}
\PY{n}{dropStmt} \PY{o}{=} \PY{n}{ibm\PYZus{}db}\PY{o}{.}\PY{n}{exec\PYZus{}immediate}\PY{p}{(}\PY{n}{conn}\PY{p}{,} \PY{n}{dropQuery}\PY{p}{)}
\end{Verbatim}
\end{tcolorbox}

    \begin{Verbatim}[commandchars=\\\{\}]

        ---------------------------------------------------------------------------

        Exception                                 Traceback (most recent call last)

        <ipython-input-7-83413676a2ca> in <module>
          3 
          4 \#Now execute the drop statment
    ----> 5 dropStmt = ibm\_db.exec\_immediate(conn, dropQuery)
    

        Exception: [IBM][CLI Driver][DB2/LINUXX8664] SQL0204N  "NRX71347.INSTRUCTOR" is an undefined name.  SQLSTATE=42704 SQLCODE=-204

    \end{Verbatim}

    \hypertarget{dont-worry-if-you-get-this-error}{%
\subsection{Dont worry if you get this
error:}\label{dont-worry-if-you-get-this-error}}

If you see an exception/error similar to the following, indicating that
INSTRUCTOR is an undefined name, that's okay. It just implies that the
INSTRUCTOR table does not exist in the table - which would be the case
if you had not created it previously.

Exception: {[}IBM{]}{[}CLI Driver{]}{[}DB2/LINUXX8664{]} SQL0204N
``ABC12345.INSTRUCTOR'' is an undefined name. SQLSTATE=42704
SQLCODE=-204

    \begin{tcolorbox}[breakable, size=fbox, boxrule=1pt, pad at break*=1mm,colback=cellbackground, colframe=cellborder]
\prompt{In}{incolor}{10}{\boxspacing}
\begin{Verbatim}[commandchars=\\\{\}]
\PY{c+c1}{\PYZsh{}Construct the Create Table DDL statement \PYZhy{} replace the ... with rest of the statement}
\PY{n}{createQuery} \PY{o}{=} \PY{l+s+s2}{\PYZdq{}}\PY{l+s+s2}{create table INSTRUCTOR(id INTEGER PRIMARY KEY NOT NULL, fname VARCHAR(20), lname VARCHAR(20), city VARCHAR(20), ccode CHAR(2))}\PY{l+s+s2}{\PYZdq{}}

\PY{c+c1}{\PYZsh{}Now fill in the name of the method and execute the statement}
\PY{n}{createStmt} \PY{o}{=} \PY{n}{ibm\PYZus{}db}\PY{o}{.}\PY{n}{exec\PYZus{}immediate}\PY{p}{(}\PY{n}{conn}\PY{p}{,} \PY{n}{createQuery}\PY{p}{)}
\end{Verbatim}
\end{tcolorbox}

    Double-click \textbf{here} for the solution.

    \hypertarget{task-5-insert-data-into-the-table}{%
\subsection{Task 5: Insert data into the
table}\label{task-5-insert-data-into-the-table}}

In this step we will insert some rows of data into the table.

The INSTRUCTOR table we created in the previous step contains 3 rows of
data:

We will start by inserting just the first row of data, i.e.~for
instructor Rav Ahuja

    \begin{tcolorbox}[breakable, size=fbox, boxrule=1pt, pad at break*=1mm,colback=cellbackground, colframe=cellborder]
\prompt{In}{incolor}{11}{\boxspacing}
\begin{Verbatim}[commandchars=\\\{\}]
\PY{c+c1}{\PYZsh{}Construct the query \PYZhy{} replace ... with the insert statement}
\PY{n}{insertQuery} \PY{o}{=} \PY{l+s+s2}{\PYZdq{}}\PY{l+s+s2}{INSERT INTO INSTRUCTOR VALUES (1, }\PY{l+s+s2}{\PYZsq{}}\PY{l+s+s2}{Rav}\PY{l+s+s2}{\PYZsq{}}\PY{l+s+s2}{, }\PY{l+s+s2}{\PYZsq{}}\PY{l+s+s2}{Ahuja}\PY{l+s+s2}{\PYZsq{}}\PY{l+s+s2}{, }\PY{l+s+s2}{\PYZsq{}}\PY{l+s+s2}{TORONTO}\PY{l+s+s2}{\PYZsq{}}\PY{l+s+s2}{, }\PY{l+s+s2}{\PYZsq{}}\PY{l+s+s2}{CA}\PY{l+s+s2}{\PYZsq{}}\PY{l+s+s2}{)}\PY{l+s+s2}{\PYZdq{}}

\PY{c+c1}{\PYZsh{}execute the insert statement}
\PY{n}{insertStmt} \PY{o}{=} \PY{n}{ibm\PYZus{}db}\PY{o}{.}\PY{n}{exec\PYZus{}immediate}\PY{p}{(}\PY{n}{conn}\PY{p}{,} \PY{n}{insertQuery}\PY{p}{)}
\end{Verbatim}
\end{tcolorbox}

    Double-click \textbf{here} for the solution.

    Now use a single query to insert the remaining two rows of data

    \begin{tcolorbox}[breakable, size=fbox, boxrule=1pt, pad at break*=1mm,colback=cellbackground, colframe=cellborder]
\prompt{In}{incolor}{12}{\boxspacing}
\begin{Verbatim}[commandchars=\\\{\}]
\PY{c+c1}{\PYZsh{}replace ... with the insert statement that inerts the remaining two rows of data}
\PY{n}{insertQuery2} \PY{o}{=} \PY{l+s+s2}{\PYZdq{}}\PY{l+s+s2}{INSERT INTO INSTRUCTOR VALUES (2, }\PY{l+s+s2}{\PYZsq{}}\PY{l+s+s2}{Raul}\PY{l+s+s2}{\PYZsq{}}\PY{l+s+s2}{, }\PY{l+s+s2}{\PYZsq{}}\PY{l+s+s2}{Chong}\PY{l+s+s2}{\PYZsq{}}\PY{l+s+s2}{, }\PY{l+s+s2}{\PYZsq{}}\PY{l+s+s2}{Markham}\PY{l+s+s2}{\PYZsq{}}\PY{l+s+s2}{, }\PY{l+s+s2}{\PYZsq{}}\PY{l+s+s2}{CA}\PY{l+s+s2}{\PYZsq{}}\PY{l+s+s2}{), (3, }\PY{l+s+s2}{\PYZsq{}}\PY{l+s+s2}{Hima}\PY{l+s+s2}{\PYZsq{}}\PY{l+s+s2}{, }\PY{l+s+s2}{\PYZsq{}}\PY{l+s+s2}{Vasudevan}\PY{l+s+s2}{\PYZsq{}}\PY{l+s+s2}{, }\PY{l+s+s2}{\PYZsq{}}\PY{l+s+s2}{Chicago}\PY{l+s+s2}{\PYZsq{}}\PY{l+s+s2}{, }\PY{l+s+s2}{\PYZsq{}}\PY{l+s+s2}{US}\PY{l+s+s2}{\PYZsq{}}\PY{l+s+s2}{)}\PY{l+s+s2}{\PYZdq{}}

\PY{c+c1}{\PYZsh{}execute the statement}
\PY{n}{insertStmt2} \PY{o}{=} \PY{n}{ibm\PYZus{}db}\PY{o}{.}\PY{n}{exec\PYZus{}immediate}\PY{p}{(}\PY{n}{conn}\PY{p}{,} \PY{n}{insertQuery2}\PY{p}{)}
\end{Verbatim}
\end{tcolorbox}

    Double-click \textbf{here} for the solution.

    \hypertarget{task-6-query-data-in-the-table}{%
\subsection{Task 6: Query data in the
table}\label{task-6-query-data-in-the-table}}

In this step we will retrieve data we inserted into the INSTRUCTOR
table.

    \begin{tcolorbox}[breakable, size=fbox, boxrule=1pt, pad at break*=1mm,colback=cellbackground, colframe=cellborder]
\prompt{In}{incolor}{23}{\boxspacing}
\begin{Verbatim}[commandchars=\\\{\}]
\PY{c+c1}{\PYZsh{}Construct the query that retrieves all rows from the INSTRUCTOR table}
\PY{n}{selectQuery} \PY{o}{=} \PY{l+s+s2}{\PYZdq{}}\PY{l+s+s2}{select * from INSTRUCTOR}\PY{l+s+s2}{\PYZdq{}}

\PY{c+c1}{\PYZsh{}Execute the statement}
\PY{n}{selectStmt} \PY{o}{=} \PY{n}{ibm\PYZus{}db}\PY{o}{.}\PY{n}{exec\PYZus{}immediate}\PY{p}{(}\PY{n}{conn}\PY{p}{,} \PY{n}{selectQuery}\PY{p}{)}

\PY{c+c1}{\PYZsh{}Fetch the Dictionary (for the first row only) \PYZhy{} replace ... with your code}
\PY{n}{ibm\PYZus{}db}\PY{o}{.}\PY{n}{fetch\PYZus{}both}\PY{p}{(}\PY{n}{selectStmt}\PY{p}{)}
\end{Verbatim}
\end{tcolorbox}

            \begin{tcolorbox}[breakable, size=fbox, boxrule=.5pt, pad at break*=1mm, opacityfill=0]
\prompt{Out}{outcolor}{23}{\boxspacing}
\begin{Verbatim}[commandchars=\\\{\}]
\{'ID': 1,
 0: 1,
 'FNAME': 'Rav',
 1: 'Rav',
 'LNAME': 'Ahuja',
 2: 'Ahuja',
 'CITY': 'TORONTO',
 3: 'TORONTO',
 'CCODE': 'CA',
 4: 'CA'\}
\end{Verbatim}
\end{tcolorbox}
        
    Double-click \textbf{here} for the solution.

    \begin{tcolorbox}[breakable, size=fbox, boxrule=1pt, pad at break*=1mm,colback=cellbackground, colframe=cellborder]
\prompt{In}{incolor}{24}{\boxspacing}
\begin{Verbatim}[commandchars=\\\{\}]
\PY{c+c1}{\PYZsh{}Fetch the rest of the rows and print the ID and FNAME for those rows}
\PY{k}{while} \PY{n}{ibm\PYZus{}db}\PY{o}{.}\PY{n}{fetch\PYZus{}row}\PY{p}{(}\PY{n}{selectStmt}\PY{p}{)} \PY{o}{!=} \PY{k+kc}{False}\PY{p}{:}
    \PY{n+nb}{print} \PY{p}{(}\PY{l+s+s2}{\PYZdq{}}\PY{l+s+s2}{ ID:}\PY{l+s+s2}{\PYZdq{}}\PY{p}{,}  \PY{n}{ibm\PYZus{}db}\PY{o}{.}\PY{n}{result}\PY{p}{(}\PY{n}{selectStmt}\PY{p}{,} \PY{l+m+mi}{0}\PY{p}{)}\PY{p}{,} \PY{l+s+s2}{\PYZdq{}}\PY{l+s+s2}{ FNAME:}\PY{l+s+s2}{\PYZdq{}}\PY{p}{,}  \PY{n}{ibm\PYZus{}db}\PY{o}{.}\PY{n}{result}\PY{p}{(}\PY{n}{selectStmt}\PY{p}{,} \PY{l+s+s2}{\PYZdq{}}\PY{l+s+s2}{FNAME}\PY{l+s+s2}{\PYZdq{}}\PY{p}{)}\PY{p}{)}
\end{Verbatim}
\end{tcolorbox}

    \begin{Verbatim}[commandchars=\\\{\}]
 ID: 2  FNAME: Raul
 ID: 3  FNAME: Hima
    \end{Verbatim}

    Double-click \textbf{here} for the solution.

    Bonus: now write and execute an update statement that changes the Rav's
CITY to MOOSETOWN

    \begin{tcolorbox}[breakable, size=fbox, boxrule=1pt, pad at break*=1mm,colback=cellbackground, colframe=cellborder]
\prompt{In}{incolor}{25}{\boxspacing}
\begin{Verbatim}[commandchars=\\\{\}]
\PY{c+c1}{\PYZsh{}Enter your code below}
\PY{c+c1}{\PYZsh{}Construct the query that changes the Rav\PYZsq{}s CITY to MOOSETOWN}
\PY{n}{updateQuery} \PY{o}{=} \PY{l+s+s2}{\PYZdq{}}\PY{l+s+s2}{update INSTRUCTOR set city = }\PY{l+s+s2}{\PYZsq{}}\PY{l+s+s2}{MOOSETOWN}\PY{l+s+s2}{\PYZsq{}}\PY{l+s+s2}{ where FNAME = }\PY{l+s+s2}{\PYZsq{}}\PY{l+s+s2}{Rav}\PY{l+s+s2}{\PYZsq{}}\PY{l+s+s2}{\PYZdq{}}

\PY{c+c1}{\PYZsh{}Execute the statement}
\PY{n}{updateStmt} \PY{o}{=} \PY{n}{ibm\PYZus{}db}\PY{o}{.}\PY{n}{exec\PYZus{}immediate}\PY{p}{(}\PY{n}{conn}\PY{p}{,} \PY{n}{updateQuery}\PY{p}{)}
\end{Verbatim}
\end{tcolorbox}

    Double-click \textbf{here} for the solution.

    \hypertarget{task-7-retrieve-data-into-pandas}{%
\subsection{Task 7: Retrieve data into
Pandas}\label{task-7-retrieve-data-into-pandas}}

In this step we will retrieve the contents of the INSTRUCTOR table into
a Pandas dataframe

    \begin{tcolorbox}[breakable, size=fbox, boxrule=1pt, pad at break*=1mm,colback=cellbackground, colframe=cellborder]
\prompt{In}{incolor}{26}{\boxspacing}
\begin{Verbatim}[commandchars=\\\{\}]
\PY{k+kn}{import} \PY{n+nn}{pandas}
\PY{k+kn}{import} \PY{n+nn}{ibm\PYZus{}db\PYZus{}dbi}
\end{Verbatim}
\end{tcolorbox}

    \begin{tcolorbox}[breakable, size=fbox, boxrule=1pt, pad at break*=1mm,colback=cellbackground, colframe=cellborder]
\prompt{In}{incolor}{27}{\boxspacing}
\begin{Verbatim}[commandchars=\\\{\}]
\PY{c+c1}{\PYZsh{}connection for pandas}
\PY{n}{pconn} \PY{o}{=} \PY{n}{ibm\PYZus{}db\PYZus{}dbi}\PY{o}{.}\PY{n}{Connection}\PY{p}{(}\PY{n}{conn}\PY{p}{)}
\end{Verbatim}
\end{tcolorbox}

    \begin{tcolorbox}[breakable, size=fbox, boxrule=1pt, pad at break*=1mm,colback=cellbackground, colframe=cellborder]
\prompt{In}{incolor}{28}{\boxspacing}
\begin{Verbatim}[commandchars=\\\{\}]
\PY{c+c1}{\PYZsh{}query statement to retrieve all rows in INSTRUCTOR table}
\PY{n}{selectQuery} \PY{o}{=} \PY{l+s+s2}{\PYZdq{}}\PY{l+s+s2}{select * from INSTRUCTOR}\PY{l+s+s2}{\PYZdq{}}

\PY{c+c1}{\PYZsh{}retrieve the query results into a pandas dataframe}
\PY{n}{pdf} \PY{o}{=} \PY{n}{pandas}\PY{o}{.}\PY{n}{read\PYZus{}sql}\PY{p}{(}\PY{n}{selectQuery}\PY{p}{,} \PY{n}{pconn}\PY{p}{)}

\PY{c+c1}{\PYZsh{}print just the LNAME for first row in the pandas data frame}
\PY{n}{pdf}\PY{o}{.}\PY{n}{LNAME}\PY{p}{[}\PY{l+m+mi}{0}\PY{p}{]}
\end{Verbatim}
\end{tcolorbox}

            \begin{tcolorbox}[breakable, size=fbox, boxrule=.5pt, pad at break*=1mm, opacityfill=0]
\prompt{Out}{outcolor}{28}{\boxspacing}
\begin{Verbatim}[commandchars=\\\{\}]
'Ahuja'
\end{Verbatim}
\end{tcolorbox}
        
    \begin{tcolorbox}[breakable, size=fbox, boxrule=1pt, pad at break*=1mm,colback=cellbackground, colframe=cellborder]
\prompt{In}{incolor}{29}{\boxspacing}
\begin{Verbatim}[commandchars=\\\{\}]
\PY{c+c1}{\PYZsh{}print the entire data frame}
\PY{n}{pdf}
\end{Verbatim}
\end{tcolorbox}

            \begin{tcolorbox}[breakable, size=fbox, boxrule=.5pt, pad at break*=1mm, opacityfill=0]
\prompt{Out}{outcolor}{29}{\boxspacing}
\begin{Verbatim}[commandchars=\\\{\}]
   ID FNAME      LNAME       CITY CCODE
0   1   Rav      Ahuja  MOOSETOWN    CA
1   2  Raul      Chong    Markham    CA
2   3  Hima  Vasudevan    Chicago    US
\end{Verbatim}
\end{tcolorbox}
        
    Once the data is in a Pandas dataframe, you can do the typical pandas
operations on it.

For example you can use the shape method to see how many rows and
columns are in the dataframe

    \begin{tcolorbox}[breakable, size=fbox, boxrule=1pt, pad at break*=1mm,colback=cellbackground, colframe=cellborder]
\prompt{In}{incolor}{30}{\boxspacing}
\begin{Verbatim}[commandchars=\\\{\}]
\PY{n}{pdf}\PY{o}{.}\PY{n}{shape}
\end{Verbatim}
\end{tcolorbox}

            \begin{tcolorbox}[breakable, size=fbox, boxrule=.5pt, pad at break*=1mm, opacityfill=0]
\prompt{Out}{outcolor}{30}{\boxspacing}
\begin{Verbatim}[commandchars=\\\{\}]
(3, 5)
\end{Verbatim}
\end{tcolorbox}
        
    \hypertarget{task-8-close-the-connection}{%
\subsection{Task 8: Close the
Connection}\label{task-8-close-the-connection}}

We free all resources by closing the connection. Remember that it is
always important to close connections so that we can avoid unused
connections taking up resources.

    \begin{tcolorbox}[breakable, size=fbox, boxrule=1pt, pad at break*=1mm,colback=cellbackground, colframe=cellborder]
\prompt{In}{incolor}{31}{\boxspacing}
\begin{Verbatim}[commandchars=\\\{\}]
\PY{n}{ibm\PYZus{}db}\PY{o}{.}\PY{n}{close}\PY{p}{(}\PY{n}{conn}\PY{p}{)}
\end{Verbatim}
\end{tcolorbox}

            \begin{tcolorbox}[breakable, size=fbox, boxrule=.5pt, pad at break*=1mm, opacityfill=0]
\prompt{Out}{outcolor}{31}{\boxspacing}
\begin{Verbatim}[commandchars=\\\{\}]
True
\end{Verbatim}
\end{tcolorbox}
        
    \hypertarget{summary}{%
\subsection{Summary}\label{summary}}

In this tutorial you established a connection to a database instance of
DB2 Warehouse on Cloud from a Python notebook using ibm\_db API. Then
created a table and insert a few rows of data into it. Then queried the
data. You also retrieved the data into a pandas dataframe.

    Copyright © 2017-2018
\href{cognitiveclass.ai?utm_source=bducopyrightlink\&utm_medium=dswb\&utm_campaign=bdu}{cognitiveclass.ai}.
This notebook and its source code are released under the terms of the
\href{https://bigdatauniversity.com/mit-license/}{MIT License}.


    % Add a bibliography block to the postdoc
    
    
    
\end{document}
